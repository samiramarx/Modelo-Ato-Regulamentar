\documentclass[a4paper,11pt]{report} 
\usepackage[utf8]{inputenc} % Permite caracteres acentuados
\setlength{\parindent}{0pt} % Retira indentação dos parágrafos
\usepackage{enumitem}
\renewcommand*{\theenumi}{\thesection.\arabic{enumi}} % Permite numeração continuada a partir dos capítulos e seções.
\renewcommand*{\theenumii}{\theenumi.\arabic{enumii}} % Permite numeração continuada a partir dos capítulos e seções.
\usepackage{hyperref} % Utilizado no sistema de referência cruzada - vide arquivo "cross-reference.txt" 
\usepackage{enumitem} % to control layout of itemize, enumerate, description
\usepackage[portuguese]{babel}
\usepackage{fancyhdr}
\usepackage{lipsum}
\pagestyle{fancy}
\setlength{\headheight}{26pt}
\lhead{Secretaria de Estado de Transportes e Obras Públicas\\
Subsecretaria de Regulação de Transportes}

\setlength\parskip{2ex} % Inclui espaço entre parágrafos

%-----------------------------------------------------------
%	PREENCHIMENTO DE DADOS
%-----------------------------------------------------------

% Dados do Ato
\newcommand{\NumeroAto}{$\bullet$}
\newcommand{\DataAssinatura}{\today}
\newcommand{\Signatario}{\textit{[Nome Signatário]}}
\newcommand{\CargoSignatario}{Subsecretário de Regulação Transportes}


%------------------------------------------------------------
\begin{document}
%------------------------------------------------------------

\begin{center}
ATO REGULAMENTAR \NumeroAto, de \DataAssinatura
\end{center}

\leftskip12em
\textbf{\lipsum[1]}

\leftskip0em
O \CargoSignatario, no uso de suas atribuições e considerando a Lei Delegada 128/2007 e o Decreto Estadual 44.603/2007,

RESOLVE:

\begin{enumerate}[label=Art. \arabic*]
\item \label{itm:A5JT} Serão sumariamente indeferidos, pela Superintendência de Transporte Intermunicipal, os pleitos operacionais dirigidos à STI que se enquadrarem nas seguintes situações:

\begin{enumerate}[label=\roman*.]
\item \label{itm:3CK8} Pedidos de alteração de regime de funcionamento de linhas desacompanhados dos seguintes documentos:

a. QRF (s) vigentes da linha em questão;
b. QRF(s) modelos com as modificações pretendidas;
c. Croqui, devidamente visado pela CRG envolvida, figurando as padronizações de quilometragem, por tipo de piso e de tempo de percurso entre os pontos de seccionamento nas situações vigente e pretendida. Para alteração de horários não é necessária a apresentação e croqui.

\item \label{itm:6FZL} Pedidos de alteração de regime de funcionamento de linhas com menos de 45 (quarenta e cinco) dias da data de expedição do QRF a ser estudado.
\item \label{itm:3QJD} Pedidos, de conteúdos tecnicamente semelhantes, já indeferidos, sem ocorrência de fato novo.
\item \label{itm:WHLA} Pedidos envolvendo assuntos múltiplos não correlativos.
\item \label{itm:6WF3} Pedido de alteração de Regime de Funcionamento de Linha quando existe processo em tramitação, desta linha, sem decisão.
\item \label{itm:EUAC} Retificação de pedidos por mais de uma vez durante os estudos de alteração do QRF da linha.
\item \label{itm:NFZG} Pedidos que contrariam o Decreto Estadual nº 44.603/2007.
\end{enumerate}

\item \label{itm:DS3Z} Os pedidos, que não se enquadrarem no item 1, serão estudados na forma regulamentar.

Parágrafo único - Independentemente dos estudos citados no caput deste artigo, a Área Técnica poderá propor ao Superintendente de Transporte Intermunicipal, indeferimento sumário, face o desnecessário ou injustificável prosseguimento regulamentar do pleito.

\item \label{itm:N2VY} Somente serão analisados os pedidos de revisão de Decisão do Subsecretário de Transportes,quando protocolados no prazo máximo de 10 (dez) dia dias a contar da publicação do Ato no "Minas Gerais".
\begin{enumerate}[label= \S \arabic*]

\item Serão cobrados da delegatária responsável pela solicitação da revisão, os valores referentes às despesas com publicação no "Minas Gerais".

\item Este é outro parágrafo.
\end{enumerate}

\item \label{itm:9REB} Esta Decisão entra em vigor na data de sua publicação.

\end{enumerate}

\begin{center}
\Signatario

\CargoSignatario
\end{center}
%----------------------------------------------------------
\end{document}
